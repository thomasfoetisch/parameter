\section{Introduction}
Ce document d\'ecrit le langage de d\'efinition de param\`etres
accept\'e par le parseur du package \texttt{parameter}. Un fichier de
param\`etre permet de sp\'ecifier un ensemble de cl\'es-valeurs, ou
les cl\'es sont des cha\^ines de caract\`eres et les valeurs
appartiennent \`a l'un des cinques types disponibles. On appellera
``set de param\`etres'' un ensemble de cl\'es-valeurs.

De plus, un fichier de param\`etre permet de d\'efinir non pas un
unique set de param\`etres, mais une collection de set de
param\`etres, \`a travers de laquelle le code client pourra it\'erer.


La section \ref{sec:syntax} pr\'esente la syntaxe formelle accept\'ee
par le parseur ainsi que la d\'efinition des symboles terminaux. Cette
section constitue une r\'ef\'erence concernant la syntaxe d'un fichier
de param\`etres. La section \ref{sec:semantic} d\'emontre l'usage et
la s\'emantique des diff\'erentes constructions syntaxiques \`a
travers une s\'erie d'exemples. Finalement, la section \ref{sec:api}
documente l'API \texttt{C++} qui permet de lire des fichiers et
d'acc\'eder aux valeurs des param\`etres sp\'ecifi\'ees de chaque
\'element de la collection r\'esultante.


\section{Syntaxe formelle}\label{sec:syntax}
Dans cette partie, on d\'efinit de fa\c con formelle la syntaxe
accept\'ee par le lexer et le parser. Les informations pr\'esent\'ees
dans cette partie tienne lieu de r\'eference, apr\`es le code
lui-m\^eme en ce qui concerne la syntaxe et s\'emantique du
langage.

\subsection{Grammaire}\label{sec:grammar}
La syntaxe accept\'ee par le parseur du fichier de param\`etre est une
grammaire LL(1) d\'efinie ci-dessous. Les symbols entre chevrons
correspondent aux symbols non-terminaux, les symbols sans
d\'ecorations correspondent aux symbols terminaux et les symbols
entres guillemets simples correspondent \`a leur valeur litterale. Le
symbol \synt{start} est bien entendu r\`egle de production initiale,
et $\epsilon$ correspond \`a la production vide.
\begin{grammar}
  <start> ::= <statment-list>
  
  <statment-list> ::= <statment> <statment-list> \alt $\epsilon$
  
  <statment> ::= <inclusion> \alt <parameter-definition> \alt <group-definition>
  
  <inclusion> ::= 'include' literal-string

  <parameter-definition> ::= key def-symbol <literal-list> \alt 'override' key def-symbol <literal-list>

  <literal-list> ::= <literal> ',' <literal-list> \alt $\epsilon$
  
  <literal> ::= key \alt literal-string \alt literal-boolean \alt literal-integer \alt literal-real \alt enum-item

  <group-definition> ::= '[' <parameter-definition-list> ']'

  <parameter-definition-list> ::= <parameter-definition> <parameter-definition-list> \alt $\epsilon$
\end{grammar}
On notera que cette syntaxe n'utilise pas de symbol de terminaison de
ligne, et que les seuls caract\`eres de ponctuation qui apparaissent
sont la virgule \lit{,} et les crochets \lit{[} et \lit{]}. Malgr\'e tout cette syntaxe
n'est pas ambig\"ue au sens du parser, et est intuitive \`a
d\'echiffrer pour un humain.

\subsection{Symbols terminaux}\label{sec:terminals}
Les expressions r\'eguli\`eres des symbols litt\'eraux \'etant
triviales, on d\'ecrit ici uniquement les symbols terminaux:
\begin{itemize}
\item \texttt{key} = \texttt{/[-\_a-zA-Z0-9]+/},
\item \texttt{literal-boolean} = \texttt{/(true)|(false)|(yes)|(no)|(on)|(off)/},
\item \texttt{literal-string} = \texttt{/"($\lbrack$\^{}"\textbackslash\textbackslash$\rbrack$|(\textbackslash\textbackslash")|(\textbackslash\textbackslash\textbackslash\textbackslash))*"/},
\item \texttt{literal-real} = \texttt{/[+-]?((\textbackslash.\textbackslash d+)|(\textbackslash d+\textbackslash.)|(\textbackslash d+\textbackslash.\textbackslash d+)|(\textbackslash d+))(\lbrack eE\rbrack\lbrack +-\rbrack?\textbackslash d+)?/},
\item \texttt{literal-integer} = \texttt{/\lbrack+-\rbrack?\textbackslash d+/},
\item \texttt{enum-item} = \texttt{/\#\lbrack-\_a-zA-Z0-9\rbrack+/},
\item \texttt{def-symbol} = \texttt{/=|:|(->)/}.
\end{itemize}
Finalement, chaque token doit \^etre s\'epar\'e par une s\'equence de
caract\`eres qui correspond \`a l'expression r\'eguli\`ere
\texttt{/(\textbackslash s|(;\lbrack\^{}\textbackslash%
n]*\textbackslash n))*/}, ce qui implique que les terminaisons de
lignes pr\'ec\'ed\'ees par le caract\`ere point-virgule '\texttt{;}'
sont ignor\'ees et permettent l'insertion de commentaires.


\section{Exemples et s\'emantique}\label{sec:semantic}

\subsection{Commentaires}
Comme indiqu\'e ci-dessus, des commentaires peuvent \^etre ins\'er\'es
dans une liste de param\`etres. Un commentaire commence par le
caract\`ere '\texttt{;}' et se termine par le retour \`a la ligne. Il
n'y a pas de notion de commentaire multiligne: chaque nouvelle ligne
de commentaire doit \^etre pr\'ec\'ed\'ee par un '\texttt{;}'. Par
exemple:
\begin{lstlisting}[language={},frame=single,basicstyle=\ttfamily]
  ; Ceci est un commentaire,
  ; ceci est une deuxieme ligne de commentaire.
  
  var = "str"   ; Definition d'un parametre
\end{lstlisting}
On utilisera les commentaires pour d\'ecrire la s\'emantique d'un
param\`etre, ou pour donner des pr\'ecisions sur la valeur
sp\'ecifi\'ee, comme les unit\'es physiques, par exemples.

On \'evitera d'utiliser des commentaires pour ``d\'esactiver'' une ou
des d\'efinitions, en revanche. Plusieurs construction dans la
grammaire sont pr\'evues pour r\'epondre \`a ce besoin, l'une d'elles
\'etant le sujet de la partie ci-dessous.


\subsection{Inclusion}
Souvent, plusieurs applications partagent des m\^emes groupes de
param\`etres. Par exemple, on peut imaginer un groupe de param\`etre
qui encode les propri\'et\'es thermique de certains mat\'eriaux, la
description d'une g\'eometrie, ou encore des param\`etres
sp\'ecifiques \`a une certaine m\'ethode num\'erique.

Afin de minimiser la duplication du code, la syntaxe permet
l'inclusion du fichiers de param\`etres, par l'interm\'ediaire du
mot-cl\'e \texttt{include}. Supposons que le fichier \texttt{water.conf}
existe et contienne:
\begin{lstlisting}[language={},frame=single,basicstyle=\ttfamily]
  density = 1.0                 ; [g/cm^3]
  specific-heat = 4.182         ; [j/g/K]
  dynamical-viscosity = 2.0e-3  ; [Pa s]
\end{lstlisting}
On peut ensuite r\'eutiliser ces d\'efinitions dans un nouveau fichier
de param\`etre \texttt{driven-cavity.conf}:
\begin{lstlisting}[language={},frame=single,basicstyle=\ttfamily]
  include "water.conf"

  box-size = 1.0 ; [m]
  boundary-velocity = 0.15 ; [m/s]
\end{lstlisting}
Le chemin d'un fichier est toujours relatif au fichier qui
l'inclu. Ici, il faut que le fichier \texttt{water.conf} doit se
trouver dans le m\^eme dossier que le \texttt{driven-cavity.conf}. Le
chemin d'un fichier inclu ne peut pas \^etre absolu.


\subsection{Definition de param\`etre}
Un param\`etre est d\'efini par l'association entre un identifiant
(\texttt{key} tel que d\'efini dans la grammaire) et une valeur. Une
valeur appartient \`a l'un des cinq types pr\'edefinis:
\texttt{string}, \texttt{integer},\texttt{real}, \texttt{boolean} et
\texttt{enumeration}. Dans l'exemple suivant, on d\'efinit cinq
param\`etres, un de chaque type, avec un commentaire associ\'e:
\begin{lstlisting}[language={},frame=single,basicstyle=\ttfamily]
  string-parameter = "Hello, world"  ; The classical example

  int-parameter = -1234              ; The opposite of the
                                     ;  natural number 1234

  real-parameter = -3.1415e+00       ; The negative pi number

  boolean-parameter = yes            ; A true value. Equivalent
                                     ;  to 'true' and 'on'.

  enum-parameter = #neumann          ; enum-parameter has value
                                     ;  'neumann', which the user
                                     ;  will have to validate

  str = string-parameter             ; str has the same value as
                                     ;  string-parameter definied
                                     ;  above
\end{lstlisting}
Vous pouvez vous r\'ef\'erer \`a la partie \ref{sec:terminals} pour
trouver une r\'ef\'erence de la syntaxe de chaque partie de l'exemple.

On remarque dans l'exemple ci-dessus qu'on peut d\'efinir un
param\`etre \`a partir d'un autre, \texttt{str} dans ce cas se verra
associer la valeur de \texttt{string-parameter}, soit la cha\^ine
\texttt{"Hello, world"}.

\paragraph{Remarque} Le type du param\`etre \texttt{int-parameter} est
\texttt{integer}, tandis que le type du param\`etre
\texttt{real-parameter} est \texttt{real}. Les nombres entiers et les
nombres flottant ne sont pas \'equivalent et ne sont pas
interchangeable. Si l'API \texttt{C++} demande un \texttt{integer}, on
ne peu pas sp\'ecifier un \texttt{real}, et reciproquement.

\paragraph{Remarque} On choisira de pr\'ef\'erence des noms de
param\`etres descriptifs, en caract\`eres minuscule, et les mots
s\'epar\'es par des trait-d'unions. Le nom d'un param\`etre constitue
une partie, sinon toute la documentation qui le concerne, il importe
donc de choisir soigneusement le noms des param\`etres. En
particulier, on \'evitera les abbr\'eviations, les identifiants de
moins de 5 caract\`eres, etc.

\subsection{Collection de param\`etres}
Il arrive souvent en analyse num\'erique de devoir lancer une s\'erie
de calculs pour lequels un unique param\`etre varie. Un exemple
classique est l'\'etude de convergence d'un sch\`ema num\'erique. Lors
d'une \'etude de convergence, on cherche \`a ex\'ecuter l'algorithme
pour un nombre pr\'ed\'efini de la tailles de subdivisions en espace,
par exemple.

La syntaxe et s\'emantique de d\'efinition de param\`etre pr\'evoit ce
cas de figure, et permet de d\'efinir une collection de sets de
param\`etres.

Soit \texttt{n} le nombre de subdivisions. On peut d\'efinir le
param\`etre \texttt{n} de la mani\'ere suivante, en sp\'ecifiant une
liste de valeurs:
\begin{lstlisting}[language={},frame=single,basicstyle=\ttfamily]
  n = 16, 32, 64
\end{lstlisting}
ce qui correspond \`a la collection de sets de param\`etres suivante:
\begin{equation}
  \cparent{\mathtt{n} = 16}, \cparent{\mathtt{n} = 32}, \cparent{\mathtt{n} = 64}.
\end{equation}
Le code se chargera alors d'executer l'algorithme pour chacun des sets
de param\`etres.

Si des listes de valeurs sont sp\'ecifi\'ees pour plusieurs
param\`etres simultan\'ement, la collection de sets de param\`etres
r\'esultant correspond au produit cart\'esien de chaque liste. Par
exemple, si les param\`etres \texttt{n} et \texttt{m} sont d\'efinis
par:
\begin{lstlisting}[language={},frame=single,basicstyle=\ttfamily]
  m = 100, 200, 400
  n = 16, 32, 64
\end{lstlisting}
la collection de set de param\`etres r\'esultant est la suivante:
\begin{align*}
  \cparent{\mathtt{m} = 100, \mathtt{n} = 16}, \cparent{\mathtt{m} =
    100, \mathtt{n} = 32}, \cparent{\mathtt{m} = 100, \mathtt{n} =
    64},\\
  \cparent{\mathtt{m} = 200, \mathtt{n} = 16}, \cparent{\mathtt{m} =
      200, \mathtt{n} = 32}, \cparent{\mathtt{m} = 200, \mathtt{n} =
      64},\\
  \cparent{\mathtt{m} = 400, \mathtt{n} = 16}, \cparent{\mathtt{m} = 400, \mathtt{n} = 32}, \cparent{\mathtt{m} = 400, \mathtt{n} = 64}.
\end{align*}
L'ordre dans lequel le produit cart\'esien est effectu\'e n'est pas
sp\'ecifi\'e, et donc l'ordre dans lequel sera effectu\'e chaque
calcul est impr\'evisible. Par contre, l'ordre est d\'eterministe, et
sera le m\^eme d'une execution \`a l'autre.

Bien entendu, il y a des situations o\`u ce n'est pas le comportement
souhait\'e, et ou l'on aimerait qque chaque liste de param\`etres soit group\'ee
s\'equentiellement. La section suivante traite de ce cas de figure.

\subsection{Groupe de param\`etres}
Il y a des situation o\`u le produit cart\'esien n'est pas la bonne
approche pour d\'ecrire une collection de sets de param\`etres. Pour
reprendre l'exemple de l'etude de convergence d'un sch\'ema
spatio-temporel, on veut sp\'ecifier une subdivision spatiale li\'ee
\`a la subdivision temporelle. Si note \texttt{space-sub} et
\texttt{time-sub} le nombre de subdivisions spatiale et temporelle, on
veut par exemple construire le set de parametres:
\begin{align*}
  \cparent{\mathtt{space\mhyphen sub} = 100, \mathtt{time\mhyphen sub} = 16},\\
  \cparent{\mathtt{space\mhyphen sub} = 200, \mathtt{time\mhyphen sub} = 32},\\
  \cparent{\mathtt{space\mhyphen sub} = 400, \mathtt{time\mhyphen sub} = 64}.\\
\end{align*}
Pour ce faire, on utilise un groupe de param\`etres, group\'e par des
crochets '\texttt{[}' et '\texttt{]}':
\begin{lstlisting}[language={},frame=single,basicstyle=\ttfamily]
  [
    space-sub = 100, 200, 400
    time-sub  = 16, 32, 64
  ]
\end{lstlisting}
Il faut bien entendu que toutes les listes dans un groupe comportent
le m\^eme nombre d'\'el\'ements. Un groupe peut comportent autant de
param\`etres que n\'ecessaire, et on peut bien entendu d\'efinir des
param\`etres suppl\'ementaires avant et apr\`es l'occurence du
groupe. Finalement, plusieurs groupe peuvent \^etres d\'efinis dans un
m\^eme fichier de param\`etres. L'exemple suivant d\'emontre une
situation compl\`ete:
\begin{lstlisting}[language={},frame=single,basicstyle=\ttfamily]
  a = 1
  [
    b = 2, 3
    c = 4, 5
  ]
  [
    d = 6, 7
    e = 8, 9
  ]
\end{lstlisting}
g\'en\`ere la collection de sets de param\`etres suivant:
\begin{align*}
  \cparent{\mathtt{a} = 1, \mathtt{b} = 2, \mathtt{c} = 4, \mathtt{d} = 6, \mathtt{e} = 8},\\
  \cparent{\mathtt{a} = 1, \mathtt{b} = 3, \mathtt{c} = 5, \mathtt{d} = 6, \mathtt{e} = 8},\\
  \cparent{\mathtt{a} = 1, \mathtt{b} = 2, \mathtt{c} = 4, \mathtt{d} = 7, \mathtt{e} = 9},\\
  \cparent{\mathtt{a} = 1, \mathtt{b} = 3, \mathtt{c} = 5, \mathtt{d} = 7, \mathtt{e} = 9}.
\end{align*}
On remarque que chaque groupe participe au produit cart\'esien de
l'ensemble des param\`etres, mais pas les listes constituant chacun
des groupes.

\paragraph{Remarque} La syntaxe n'emp\^eche pas de d\'eclarer une liste
de valeur dont les types sont h\'et\'erog\`enes. Ceci dit, une
telle d\'efinition n'as pas beaucoup de sens d'un point du vue de
l'interface \texttt{C++}, puisque le type du param\`etre concern\`e
pourrait potentiellement changer dans chaque set de la collection
engendr\'ee.

\subsection{Red\'efinition de param\`etres}
On peut red\'efinir un param\`etre avec une nouvelle valeur, bien que,
telle quelle, cette pratique soit d\'ecourag\'ee. En particulier, un
warning est \'emis dans le cas suivant:
\begin{lstlisting}[language={},frame=single,basicstyle=\ttfamily]
  a = 1
  a = 2
\end{lstlisting}
Ce comportement permet me mettre en \'evidence une une erreur qui est
typiquement li\'ee \`a une faute de frappe o\`u a une faute
d'inattention.

Si le comportement souhait\'e est effectivement de red\'efinir un
param\`etre, il faut pr\'ec\'eder la d\'efinition par le mot-cl\'e
\texttt{override}, comme dans l'exemple suivant:
\begin{lstlisting}[language={},frame=single,basicstyle=\ttfamily]
  a = 1
  override a = 2
\end{lstlisting}
Si en revanche la variable qui est red\'efinie avec le mot-cl\'e
\texttt{override} n'a pas \'et\'e d\'efinie, une exception est
lev\'ee, comme dans le cas suivant:
\begin{lstlisting}[language={},frame=single,basicstyle=\ttfamily]
  a = 1
  override b = 2
\end{lstlisting}
De nouveau, ce comportement permet d'\'eviter une classe d'erreurs
li\'ees \`a la volont\'e effective de red\'efinir des param\`etres
dans certaines situations.
\paragraph{Remarque} La syntaxe n'emp\^eche pas de changer le type d'un
param\`etre par l'interm\'ediaire d'une red\'efinition. Ceci dit, une
telle red\'efinition n'as pas beaucoup de sens d'un point du vue de
l'interface \texttt{C++}.

\subsection{Interpolation de chaine de caract\`ere}
Les param\`etres dont le type est une cha\^ine de caract\`ere sont
souvent utilis\'e pour repr\'esenter des noms de fichier, ou des
fragments de nom de fichier, par exemple pour sp\'ecifier la
destination dans le syst\`eme de fichier du r\'esultat d'un calcul.

La syntaxe pr\'evoit un m\'ecanisme d'interpolation de cha\^ine de
caract\`ere, qui permet de sp\'ecifier une cha\^ine de caract\`ere
fonction des autres param\`etres d\'efinis dans le set.

La valeur d'un param\`etre peut \^etre ins\'er\'e dans une cha\^ine de
caract\`eres en sp\'ecifiant le nom de celui-ci entre accolade. Dans
l'exemple qui suit, la valeur du param\`etre \texttt{n} est
substitu\'e dans la cha\^ine de caract\`ere qui d\'efinit le
param\`etre \texttt{output}:
\begin{lstlisting}[language={},frame=single,basicstyle=\ttfamily]
  n = 100
  output = "solution-with-subdivisions-{n}.dat"
\end{lstlisting}
Dans ce cas, le param\`;etres \texttt{output} se verra assigne la
cha\^ine de caract\`ere
\texttt{"solution-with-subdivisions-100.dat"}. Bien entendu, ceci
fonctionne \'egalement avec les collections de param\`etres. Le code
suivant:
\begin{lstlisting}[language={},frame=single,basicstyle=\ttfamily]
  n = 100, 200
  output = "soln-w-sub-{n}.dat"
\end{lstlisting}
est fonctionnellement \'equivalent \`a:
\begin{lstlisting}[language={},frame=single,basicstyle=\ttfamily]
  [
    n = 100, 200
    output = "soln-w-sub-100.dat", "soln-w-sub-200.dat"
  ]
\end{lstlisting}
tout en \'etant plus court \`a \'ecrire, et moins susceptible \`a
l'introduction d'erreur.



\subsection{Exemple complet}

Pour conclure cette documentation, on donne un exemple complet
inspir\'e du code de simulation de formation de gel en p\'eriph\'erie
des particules d'alumine. Pour \'eviter de surcharger la
pr\'esentation, on a omis l'ensemble des commentaires, \`a l'exception
des unit\'es physiques.

Le fichier de configuration principal
\path{\texttt{cryolite-particle-remelt.conf}} contient les d\'efinitions
suivantes:
\begin{lstlisting}[language={},frame=single,basicstyle=\ttfamily]
  import "config/physical/cryolite-su.conf"
  import "config/model/cryolite-particle-remelt-su.conf"
  import "config/numerical/cryolite-particle-remelt.conf"
  
  output-prefix = "cryolite-particle-remelt/output"
  output-transient-solution = no
  output-final-solution = yes
  output-transition = yes
  output-beta-function = no
  output-neumann-exact-solution = no
\end{lstlisting}
Le fichier principal inclu trois fichiers de configurations qui
regroupent respectivement les param\`etres sp\'ecifiques \`a la
physique des mat\'eriaux, c'est-\`a-dire les param\`etres
thermocin\'etiques de l'alumine et du bain dans ce cas-ci, les
param\`etres sp\'ecifiques au mod\`ele, soit la sp\'ecification de la
g\'eom\'etrie, les conditions de bords, les conditions initiales, etc,
et finalement les param\`etres sp\'ecifiques \`a la m\'ethode
num\'`erique. On a \`egalement plac\'e ici tous les param\`etres qui
contr\^ole l'output de la simulation, et qui sont susceptibles de
changer fr\`equemment.

Le contenu des trois fichiers inclus est donn\'e ci-dessous. Le
contenu de \texttt{config/physical/cryolite-\-su.conf} est:
\begin{lstlisting}[language={},frame=single,basicstyle=\ttfamily]
  electrolyte-density = 2130.0e-9 ; [kg/mm^3]
  alumina-density = 2130.0e-9 ; [kg/mm^3]
  
  electrolyte-sl-low-t = 950.0 ; [C]
  electrolyte-sl-high-t = 950.0 ; [C]
  electrolyte-sl-latent-heat = 5.5083e5 ; [j/kg]
  
  solid-electrolyte-heat-capacity = 1403. ; [j/kg/K]
  liquid-electrolyte-heat-capacity = 1861.3 ; [j/kg/K]
  alumina-heat-capacity = 1403. ; [j/kg/K]
  
  solid-electrolyte-diffusivity-coefficient = 2.0e-3 ; [j/s/mm/K]
  liquid-electrolyte-diffusivity-coefficient = 2.0e-3 ; [j/s/mm/K]
  alumina-diffusivity-coefficient = 2.0e-3 ; [j/s/mm/K]
\end{lstlisting}

Le contenu de \path{\texttt{config/model/cryolite-\-particle-\-remelt-\-su.conf}} est:
\begin{lstlisting}[language={},frame=single,basicstyle=\ttfamily]
  time-end    = 0.130  ; [s]
  domain-size = .5  ; [mm]
  
  particle-radius = 0.05  ; [mm]
  
  particle-initial-temperature    = 150.0 ; [C]
  electrolyte-initial-temperature = 955.0 ; [C]
  
  left-bc-type   = #neumann
  left-bc-value  = 0.0 ; [K/mm]
  
  right-bc-type  = #dirichlet
  right-bc-value = 955.0 ; [C]
  
  coordinates = #spherical
\end{lstlisting}

Le contenu de \texttt{config/numerical/cryolite-\-particle-\-remelt.conf} est:
\begin{lstlisting}[language={},frame=single,basicstyle=\ttfamily]
  space-subdivisions = 4000
  time-subdivisions  = 4000
\end{lstlisting}

Le fichier \texttt{cryolite-particle-remelt.conf} et les trois fichier
de param\`etres inclus contitues la base de param\`etres par d\'efaut
du mod\`ele simul\'e par le code, dans ce cas la formation de gel
autour d'une particule d'alumine dans un bain \'electrolytique en
\`elements finits $\mathbb P_1-\mathbb P_1$ sous forme enthalpique
avec la formule de Chernoff.

Toutes les \'etudes de ce mod\`ele devraient \^etre impl\'ement\'ees
sous forme d'un fichier de param\`etres qui inclu la base
\texttt{cryolite-particle-remelt.conf}, puis red\'efinit certain de
ces param\`etres.

Consid\'erons le cas d'une \'etude de convergence, imp\'ement\'e par
le fichier de param\`etres
\texttt{cryolite-\-particle-\-remelt-\-conv-\-study.conf} donn\'e ci-dessous:
\begin{lstlisting}[language={},frame=single,basicstyle=\ttfamily]
  ; Convergence study, expected order in dt and h is approximatly 0.75
  
  include "cryolite-particle-remelt.conf"
  
  override-output-final-solution = yes
  override output-transition = no
  override output-beta-function = no
  override output-neumann-exact-solution = yes

  override time-end = 0.5  ; [s]
  override domain-size = 1000.0 ; [mm]
  [
    override space-subdivisions = 100, 200, 400, 800, 1600
    override time-subdivisions = 128, 256, 512, 1024, 2048
  ]

  override output-prefix
    = "cryolite-particle-remelt/conv/output-h-{space-subdivisions}"
\end{lstlisting}

On notera l'usage du mot-cl\'e \texttt{override} qui garantit que les
param\`etres que l'on red\'efinit existent au pr\'ealable.
On notera \'egalement l'usage d'un groupe de param\`etres pour
d\'efinir le set de raffinements successifs. Et finalement, on
remarquera que l'output est redirig\'e vers un dossier sp\'ecifique
pour chaque raffinement. Pour \'eviter que la ligne soit trop longue,
elle est r\'epartie sur deux lignes.

\section{Interface \texttt{C++}}\label{sec:api}
